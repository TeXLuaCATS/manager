% language=uk

\environment luatex-style

\startcomponent luatex-lua

\startchapter[reference=lua,title={Using \LUATEX}]

\startsection[title={Initialization},reference=init]

\startsubsection[title={\LUATEX\ as a \LUA\ interpreter}]

\topicindex {initialization}
\topicindex {\LUA+interpreter}

There are some situations that make \LUATEX\ behave like a standalone \LUA\
interpreter:

\startitemize[packed]
\startitem
    if a \type {--luaonly} option is given on the commandline, or
\stopitem
\startitem
    if the executable is named \type {texlua} or \type {luatexlua}, or
\stopitem
\startitem
    if the only non|-|option argument (file) on the commandline has the extension
    \type {lua} or \type {luc}.
\stopitem
\stopitemize

In this mode, it will set \LUA's \type {arg[0]} to the found script name, pushing
preceding options in negative values and the rest of the command line in the
positive values, just like the \LUA\ interpreter.

\LUATEX\ will exit immediately after executing the specified \LUA\ script and is,
in effect, a somewhat bulky stand alone \LUA\ interpreter with a bunch of extra
preloaded libraries.

\stopsubsection

\startsubsection[title={\LUATEX\ as a \LUA\ byte compiler}]

\topicindex {\LUA+byte code}

There are two situations that make \LUATEX\ behave like the \LUA\ byte compiler:

\startitemize[packed]
\startitem if a \type {--luaconly} option is given on the command line, or \stopitem
\startitem if the executable is named \type {texluac} \stopitem
\stopitemize

In this mode, \LUATEX\ is exactly like \type {luac} from the stand alone \LUA\
distribution, except that it does not have the \type {-l} switch, and that it
accepts (but ignores) the \type {--luaconly} switch. The current version of \LUA\
can dump bytecode using \type {string.dump} so we might decide to drop this
version of \LUATEX.

\stopsubsection

\startsubsection[title={Other commandline processing}]

\topicindex {command line}

When the \LUATEX\ executable starts, it looks for the \type {--lua} command line
option. If there is no \type {--lua} option, the command line is interpreted in a
similar fashion as the other \TEX\ engines. Some options are accepted but have no
consequence. The following command|-|line options are understood:

\starttabulate[|l|p|]
\DB commandline argument                \BC explanation \NC \NR
\TB
\NC \type{--credits}                    \NC display credits and exit \NC \NR
\NC \type{--debug-format}               \NC enable format debugging \NC \NR
\NC \type{--draftmode}                  \NC switch on draft mode i.e.\ generate no output in \PDF\ mode \NC \NR
\NC \type{--[no-]check-dvi-total-pages} \NC exit when DVI exceeds 65535 pages (default: check) \NC \NR
\NC \type{--[no-]file-line-error}       \NC disable/enable \type {file:line:error} style messages \NC \NR
\NC \type{--[no-]file-line-error-style} \NC aliases of \type {--[no-]file-line-error} \NC \NR
\NC \type{--fmt=FORMAT}                 \NC load the format file \type {FORMAT} \NC\NR
\NC \type{--halt-on-error}              \NC stop processing at the first error\NC \NR
\NC \type{--help}                       \NC display help and exit \NC\NR
\NC \type{--ini}                        \NC be \type {iniluatex}, for dumping formats \NC\NR
\NC \type{--interaction=STRING}         \NC set interaction mode: \type {batchmode}, \type {nonstopmode},
                                            \type {scrollmode} or \type {errorstopmode} \NC \NR
\NC \type{--jobname=STRING}             \NC set the job name to \type {STRING} \NC \NR
\NC \type{--kpathsea-debug=NUMBER}      \NC set path searching debugging flags according to the bits of
                                           \type {NUMBER} \NC \NR
\NC \type{--lua=FILE}                   \NC load and execute a \LUA\ initialization script \NC\NR
\NC \type{--luadebug}                   \NC enable the \type{debug} library\NC\NR
\NC \type{--[no-]mktex=FMT}             \NC disable/enable \type {mktexFMT} generation with \type {FMT} is
                                            \type {tex} or \type {tfm} \NC \NR
\NC \type{--nosocket}                   \NC disable the \LUA\ socket library \NC\NR
\NC \type{--no-socket}                  \NC disable the \LUA\ socket library \NC\NR
\NC \type{--socket}                     \NC enable the \LUA\ socket library \NC\NR
\NC \type{--output-comment=STRING}      \NC use \type {STRING} for \DVI\ file comment instead of date (no
                                            effect for \PDF) \NC \NR
\NC \type{--output-directory=DIR}       \NC use \type {DIR} as the directory to write files to \NC \NR
\NC \type{--output-format=FORMAT}       \NC use \type {FORMAT} for job output; \type {FORMAT} is \type {dvi}
                                            or \type {pdf} \NC \NR
\NC \type{--progname=STRING}            \NC set the program name to \type {STRING} \NC \NR
\NC \type{--recorder}                   \NC enable filename recorder \NC \NR
\NC \type{--safer}                      \NC disable easily exploitable \LUA\ commands \NC\NR
\NC \type{--[no-]shell-escape}          \NC disable/enable system calls \NC \NR
\NC \type{--shell-restricted}           \NC restrict system calls to a list of commands given in \type
                                            {texmf.cnf} \NC \NR
\NC \type{--synctex=NUMBER}             \NC enable \type {synctex} \NC \NR
\NC \type{--utc}                        \NC use utc times when applicable \NC \NR
\NC \type{--version}                    \NC display version and exit \NC \NR
\LL
\stoptabulate

We don't support \prm {write} 18 because \type {os.execute} can do the same. It
simplifies the code and makes more write targets possible.

The value to use for \prm {jobname} is decided as follows:

\startitemize
\startitem
    If \type {--jobname} is given on the command line, its argument will be the
    value for \prm {jobname}, without any changes. The argument will not be
    used for actual input so it need not exist. The \type {--jobname} switch only
    controls the \prm {jobname} setting.
\stopitem
\startitem
    Otherwise, \prm {jobname} will be the name of the first file that is read
    from the file system, with any path components and the last extension (the
    part following the last \type {.}) stripped off.
\stopitem
\startitem
    There is an exception to the previous point: if the command line goes into
    interactive mode (by starting with a command) and there are no files input
    via \prm {everyjob} either, then the \prm {jobname} is set to \type
    {texput} as a last resort.
\stopitem
\stopitemize

The file names for output files that are generated automatically are created by
attaching the proper extension (\type {log}, \type {pdf}, etc.) to the found
\prm {jobname}. These files are created in the directory pointed to by \type
{--output-directory}, or in the current directory, if that switch is not present.
If \type{--output-directory} is not empty, its value it's copied to the
\type{TEXMF_OUTPUT_DIRECTORY} env. variable; if it's empty, the value of
\type{TEXMF_OUTPUT_DIRECTORY} is the value of the output directory.

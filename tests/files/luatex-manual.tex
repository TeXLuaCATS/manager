% language=uk

\environment luatex-style

\startcomponent luatex-preamble

\startchapter[reference=preamble,title={Preamble}]

\topicindex{nodes}
\topicindex{boxes}
\topicindex{\LUA}

This is a reference manual, not a tutorial. This means that we discuss changes
relative to traditional \TEX\ and also present new functionality. As a consequence
we will refer to concepts that we assume to be known or that might be explained
later.

The average user doesn't need to know much about what is in this manual. For
instance fonts and languages are normally dealt with in the macro package that
you use. Messing around with node lists is also often not really needed at the
user level. If you do mess around, you'd better know what you're dealing with.
Reading \quotation {The \TEX\ Book} by Donald Knuth is a good investment of time
then also because it's good to know where it all started. A more summarizing
overview is given by \quotation {\TEX\ by Topic} by Victor Eijkhout. You might
want to peek in \quotation {The \ETEX\ manual} and documentation about \PDFTEX.

But \unknown\ if you're here because of \LUA, then all you need to know is that
you can call it from within a run. The macro package that you use probably will
provide a few wrapper mechanisms but the basic \lpr {directlua} command that
does the job is:

\starttyping
\directlua{tex.print("Hi there")}
\stoptyping

You can put code between curly braces but if it's a lot you can also put it in a
file and load that file with the usual \LUA\ commands.
